%---------------------------------------------------------------
\chapter{Úvod}
%---------------------------------------------------------------
\setcounter{page}{1}

Klasické modely vyhledávání ve fulltextových dokumentech naráží na problém s hledáním synonym, případně na problém vyhledávání ve více jazycích. S růstem webových technologií a obecnou globalizací je toto přitom poměrně důležitý aspekt toho, co bychom měli od takovýchto modelů očekávat. Vzhledem ke stále narůstajícímu objemu informací na internetu a celkové globalizaci, kde široce využívané jazyky jako angličtina ztrácí své prominentní postavení, je potřeba brát v potaz nutnost vyhledávat i ve zdrojích psaných v jiných jazycích.

Příkladů využití vícejazyčného vyhledávání v dokumentech je mnoho, jak v soukromé, tak ve veřejné sféře. Ať už jsou to bussiness analytici, hledající aktuální informace z různých zemí, žurnalisté píšící o aktuálním dění v zahraničí nebo vědci a akademici zkoumající problém, který už byl řešen, ovšem v jiném jazyce, multijazyčné vyhledávání nám dává možnost rychle a efektivně řešit rozdrobenost a neprovázanost dokumentů psaných v různých jazycích.

Tato práce se zabývá metodami zpracování informací (anglicky *information retrieval*). Konkrétně se zaobírá problematikou vyhledávání v kolekcích dokumentů - od samotných modelů, které se tímto zabývají, po indexaci dat a vyhodnocování úspěšnosti těchto modelů.

Dále pak zkoumá techniky zpracování přirozeného jazyka. Zejména pak *word embeddingy*, zabývající se kódováním slov do vektorové reprezentace, která je lepší pro strojové zpracování. Tyto techniky jsou schopné poradit si se zachycením asociací mezi slovy a zároveň jsou schopné reprezentovat vazby mezi přeloženými slovy, které by klasickým modelům činily potíže. Jmenovitě je kladen důraz zejména na model Word2Vec, jeho možnosti a využití pro vícejazyčné překlady.

Cílem této práce je spojení *information retrieval* modelů a jejich indexačních schopností s modelem *word2vec,* který je **schopný zachytit vazby mezi přeloženými slovy, které by klasické modely nebyly schopny postihnout. Model vzniklý tímto spojením by pak byl schopen pracovat jako multijazyčný fulltextový vyhledávač. Pro učení takového modelu můžeme s výhodou využít plně a explicitně přeložené texty, jako jsou například právní nařízení a zákony Evropské Unie.
